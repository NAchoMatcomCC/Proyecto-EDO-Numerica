%===================================================================================
% JORNADA CIENTÍFICA ESTUDIANTIL - MATCOM, UH
%===================================================================================
% Esta plantilla ha sido diseñada para ser usada en los artículos de la
% Jornada Científica Estudiantil de MatCom.
%
% Por favor, siga las instrucciones de esta plantilla y rellene en las secciones
% correspondientes.
%
% NOTA: Necesitará el archivo 'jcematcom.sty' en la misma carpeta donde esté este
%       archivo para poder utilizar esta plantila.
%===================================================================================



%===================================================================================
% PREÁMBULO
%-----------------------------------------------------------------------------------
\documentclass[a4paper,10pt,twocolumn]{article}

%===================================================================================
% Paquetes
%-----------------------------------------------------------------------------------
\usepackage{amsmath}
\usepackage{amsfonts}
\usepackage{amssymb}
\usepackage{jcematcom}
\usepackage[utf8]{inputenc}
\usepackage{listings}
\usepackage[pdftex]{hyperref}
\usepackage{caption}
\usepackage{subcaption}

%-----------------------------------------------------------------------------------
% Configuración
%-----------------------------------------------------------------------------------
\hypersetup{colorlinks,%
	    citecolor=black,%
	    filecolor=black,%
	    linkcolor=black,%
	    urlcolor=blue}

%===================================================================================



%===================================================================================
% Presentacion
%-----------------------------------------------------------------------------------
% Título
%-----------------------------------------------------------------------------------
\title{Proyecto Conjunto: Ecuaciones Diferenciales
 Ordinarias y Matemática Numérica}

%-----------------------------------------------------------------------------------
% Autores
%-----------------------------------------------------------------------------------
\author{\\
\name Anthony Ventosa Lescay \AND
\name Abner Alexandro Abreu Tamayo \AND
\name Ignacio Miguel Rodríguez Pacheco
\\
\\ \addr Grupo C212}

%-----------------------------------------------------------------------------------
% Tutores
%-----------------------------------------------------------------------------------
\tutors{\\
\\
}

%-----------------------------------------------------------------------------------
% Headings
%-----------------------------------------------------------------------------------
\jcematcomheading{\the\year}{1-\pageref{end}}{A. Uno, A. Dos}

%-----------------------------------------------------------------------------------

%===================================================================================



%===================================================================================
% DOCUMENTO
%-----------------------------------------------------------------------------------
\begin{document}

%-----------------------------------------------------------------------------------
% NO BORRAR ESTA LINEA!
%-----------------------------------------------------------------------------------
\twocolumn[
%-----------------------------------------------------------------------------------

\maketitle

%===================================================================================
% Resumen y Abstract
%-----------------------------------------------------------------------------------
\selectlanguage{spanish} % Para producir el documento en Español

%-----------------------------------------------------------------------------------
% Resumen en Español
%-----------------------------------------------------------------------------------
\begin{abstract}

	Este trabajo aborda la modelación del movimiento de una persona que desciende en paracaídas, considerando la resistencia del aire como una fuerza proporcional tanto a la velocidad como a su cuadrado, según el régimen de caída. Se establece un sistema de ecuaciones diferenciales ordinarias que describe la evolución temporal de la altura y la velocidad, incorporando un cambio en el coeficiente
    de resistencia al momento de abrir el paracaídas. El estudio incluye la identificación de los parámetros físicos relevantes, el planteamiento por tramos del modelo y la obtención de soluciones analíticas para cada etapa. Además, se analizan sistemas simplificados asociados, como la ecuación $\dot{z} = \mu - z^2$, explorando sus puntos de equilibrio, estabilidad y diagrama de bifurcación.
    Finalmente, se examina un sistema lineal que representa la caída con resistencia proporcional, destacando la existencia de velocidad terminal y el
    comportamiento asintótico de las trayectorias. El análisis conjunto permite comprender de manera integral la dinámica del descenso y sus implicaciones físicas.


\end{abstract}

%-----------------------------------------------------------------------------------
% English Abstract
%-----------------------------------------------------------------------------------
\vspace{0.5cm}

\begin{enabstract}

  This work models the motion of a parachutist descending under gravity, considering air resistance as a force proportional both to velocity and to its square, depending on the stage of the fall. A system of ordinary differential equations is formulated to describe the time evolution of height and velocity, including a change in the drag coefficient at the instant the parachute opens.
  The study presents the identification of all relevant physical parameters, the piecewise formulation of the model, and the analytical solution on each interval. Additionally, simplified associated systems are examined, such as the equation $\dot{z} = \mu - z^2$, focusing on equilibrium points, stability properties, and the corresponding bifurcation diagram. A linear system representing
  descent with linear drag is also analyzed, highlighting the existence of a terminal velocity and the asymptotic behavior of trajectories. Altogether, the analysis provides a comprehensive understanding of the dynamics of the fall and its physical implications.

\end{enabstract}

%-----------------------------------------------------------------------------------
% Palabras clave
%-----------------------------------------------------------------------------------
\begin{keywords}
	Modelación matemática, Resistencia del aire, Sistemas dinámicos, Velocidad terminal, Bifurcación.
\end{keywords}

%-----------------------------------------------------------------------------------
% Temas
%-----------------------------------------------------------------------------------
\begin{topics}
	Tema 14: Resistencia proporcional a la velocidad y a su cuadrado.
\end{topics}


%-----------------------------------------------------------------------------------
% NO BORRAR ESTAS LINEAS!
%-----------------------------------------------------------------------------------
\vspace{0.8cm}
]
%-----------------------------------------------------------------------------------


%===================================================================================

%===================================================================================
% Resumen Extendido
%-----------------------------------------------------------------------------------


%===================================================================================

\section{Modelación del Problema}

\subsection*{Resumen}
Este trabajo modela el movimiento de una persona que desciende en paracaídas, considerando la resistencia del aire 
como una fuerza proporcional a la velocidad. La situación física se describe mediante una ecuación diferencial ordinaria 
(EDO) de primer orden que relaciona la velocidad del cuerpo con el tiempo. El objetivo es representar matemáticamente 
la caída, identificar los parámetros involucrados y establecer las condiciones iniciales que caracterizan el proceso.

\subsection*{Contexto físico}
Cuando un cuerpo cae bajo la acción de la gravedad, su movimiento está influenciado por dos fuerzas principales: el peso, 
que actúa hacia abajo, y la resistencia del aire, que se opone al movimiento. Si la resistencia es proporcional a la velocidad, 
el movimiento se rige por la ecuación:
\begin{equation}
m \frac{dv}{dt} = -mg - k v,
\end{equation}
donde $m$ es la masa del cuerpo, $g$ la aceleración de la gravedad y $k$ el coeficiente de resistencia. Dividiendo entre $m$ 
se obtiene:
\begin{equation}
\frac{dv}{dt} = -g - r v, \quad \text{con } r = \frac{k}{m}.
\end{equation}

\subsection*{Identificación de parámetros}
\begin{itemize}
    \item $g = 32\,ft/s^2$: aceleración de la gravedad en unidades inglesas.
    \item $r$: coeficiente de resistencia (1/s), diferente antes y después de abrir el paracaídas.
    \begin{itemize}
        \item $r_1 = 0.15$: sin paracaídas.
        \item $r_2 = 1.5$: con paracaídas.
    \end{itemize}
    \item $y(t)$: altura (ft).
    \item $v(t)$: velocidad vertical (ft/s).
    \item $t_s = 20\,s$: instante en que se abre el paracaídas.
    \item Condiciones iniciales: $y(0) = 10000\,ft$, $v(0) = 0$.
\end{itemize}

\subsection*{Representación del problema como EDO}
El sistema completo de ecuaciones diferenciales que describe el movimiento es:
\begin{equation}
\begin{cases}
\dfrac{dy}{dt} = v(t), \\[6pt]
\dfrac{dv}{dt} = -g - r(t) v(t),
\end{cases}
\end{equation}
donde
\begin{equation}
r(t) =
\begin{cases}
r_1, & 0 \le t < t_s, \\[4pt]
r_2, & t \ge t_s.
\end{cases}
\end{equation}

\subsection*{Solución general por tramos}
La solución de la ecuación diferencial se obtiene separando las variables o usando el método del factor integrante.
Para cada tramo:
\begin{align}
v(t) &= -\frac{g}{r} + \left(v_0 + \frac{g}{r}\right)e^{-r t}, \\[4pt]
y(t) &= y_0 -\frac{g}{r}t + \left(v_0 + \frac{g}{r}\right)\frac{1 - e^{-r t}}{r}.
\end{align}
Dado que $r$ cambia en $t = t_s$, el problema se resuelve en dos etapas:
\begin{enumerate}
    \item \textit{Caída libre} ($0 \le t \le t_s$): $r = r_1$.
    \item \textit{Descenso con paracaídas} ($t \ge t_s$): $r = r_2$, con condiciones iniciales $y(t_s)$ y $v(t_s)$ heredadas del primer tramo.
\end{enumerate}

\subsection*{Interpretación física}
La solución muestra que el cuerpo alcanza una velocidad terminal 
\begin{equation}
v_T = -\frac{g}{r},
\end{equation}
en cada régimen. Cuando el paracaídas se abre, el coeficiente de resistencia aumenta, reduciendo la magnitud de la velocidad terminal 
y extendiendo el tiempo total de caída. El modelo permite predecir la evolución de la velocidad y la altura con el tiempo, 
así como estimar el tiempo total de descenso hasta llegar al suelo.





%===================================================================================

\section{Análisis del Sistema: Parte B}

\subsection*{1. Puntos de Equilibrio en función de $\mu$}

Consideremos la ecuación diferencial:
\[
\frac{dz}{dt} = \mu - z^2.
\]

Los puntos de equilibrio satisfacen:
\[
\frac{dz}{dt} = 0 \quad \Rightarrow \quad \mu - z^2 = 0.
\]

Por tanto:
\[
z^2 = \mu.
\]

\begin{itemize}
    \item \textbf{Si $\mu > 0$:} existen dos puntos de equilibrio:
    \[
    z = \sqrt{\mu}, \qquad z = -\sqrt{\mu}.
    \]

    \item \textbf{Si $\mu = 0$:} existe un único punto de equilibrio:
    \[
    z = 0.
    \]

    \item \textbf{Si $\mu < 0$:} no existen puntos de equilibrio reales.
\end{itemize}

\subsection*{2. Clasificación de Estabilidad}

Para estudiar la estabilidad, derivamos el campo vectorial:
\[
\frac{d}{dz}(\mu - z^2) = -2z.
\]

Evaluamos en los puntos de equilibrio:

\begin{itemize}
    \item \textbf{Para $\mu > 0$:}
    \begin{itemize}
        \item En $z = \sqrt{\mu}$: 
        \[
        -2\sqrt{\mu} < 0 \quad \Rightarrow \quad \text{Punto estable}.
        \]
        \item En $z = -\sqrt{\mu}$:
        \[
        2\sqrt{\mu} > 0 \quad \Rightarrow \quad \text{Punto inestable}.
        \]
    \end{itemize}

    \item \textbf{Para $\mu = 0$:}
    \begin{itemize}
        \item En $z = 0$:
        \[
        0 \quad \Rightarrow \quad \text{Caso crítico}.
        \]
    \end{itemize}

\end{itemize}

\subsection*{3. Diagrama de Bifurcación}

El diagrama de bifurcación se describe mediante:

\begin{itemize}
    \item Eje horizontal: parámetro $\mu$.
    \item Eje vertical: variable $z$.
\end{itemize}

Las curvas de equilibrio son:
\[
z = \sqrt{\mu} \quad \text{(estable, línea continua)},
\]
\[
z = -\sqrt{\mu} \quad \text{(inestable, línea discontinua)}.
\]

El punto de bifurcación ocurre en:
\[
\mu = 0,
\]
y corresponde a una \textbf{bifurcación tipo nodo-silla}.

\begin{figure}[h!]
    \centering
    \includegraphics[width=0.5\textwidth]{output.png} % usa el nombre real
    \caption{Diagrama de Bifurcación.}
\end{figure}

\subsection*{4. Interpretación Física}

En el contexto de movimiento con resistencia no lineal:

\begin{itemize}
    \item \textbf{$\mu > 0$:} el sistema alcanza una \textbf{velocidad terminal estable}:
    \[
    z = \sqrt{\mu}.
    \]

    \item \textbf{$\mu = 0$:} se encuentra el umbral o transición crítica entre regímenes dinámicos.

    \item \textbf{$\mu < 0$:} no existe velocidad terminal; el comportamiento dinámico es no acotado.
\end{itemize}

La bifurcación en $\mu = 0$ marca el punto donde desaparece la posibilidad de equilibrio, separando comportamientos físicos cualitativamente distintos.


%===================================================================================


\section{Análisis del Sistema: Parte C}

\subsection*{Sistema de Ecuaciones}

El sistema considerado es:
\[
\begin{aligned}
\frac{dy}{dt} &= v, \\
\frac{dv}{dt} &= -32 - 0.5v.
\end{aligned}
\]

\subsection*{1. Cálculo de Puntos Críticos}

Los puntos críticos satisfacen:
\[
\begin{cases}
\dfrac{dy}{dt} = 0, \\
\dfrac{dv}{dt} = 0.
\end{cases}
\]

Sustituyendo:
\[
\begin{cases}
v = 0, \\
-32 - 0.5v = 0.
\end{cases}
\]

Resolviendo:
\begin{itemize}
    \item De la primera ecuación: $v = 0$.
    \item Sustituyendo en la segunda:
    \[
    -32 - 0.5(0) = -32 \neq 0.
    \]
\end{itemize}

\textbf{Conclusión:}  
El sistema presenta una contradicción ($-32 = 0$), por tanto:

\[
\boxed{\text{NO EXISTEN PUNTOS CRÍTICOS}}
\]

\subsection*{2. Análisis del Comportamiento}

La velocidad terminal se obtiene imponiendo:
\[
\frac{dv}{dt} = 0 
\quad \Rightarrow \quad 
-32 - 0.5v = 0 
\quad \Rightarrow \quad 
v = -64 \text{ ft/s}.
\]

Las líneas nulas son:
\[
\frac{dy}{dt} = 0 \quad \text{cuando} \quad v = 0,
\]
\[
\frac{dv}{dt} = 0 \quad \text{cuando} \quad v = -64.
\]

\subsection*{3. Interpretación Física}

\begin{itemize}
    \item $\frac{dy}{dt} = v$: define la velocidad vertical.
    \item $\frac{dv}{dt} = -32 - 0.5v$:
    \begin{itemize}
        \item $-32$: aceleración gravitacional constante.
        \item $-0.5v$: resistencia lineal proporcional a la velocidad.
    \end{itemize}
\end{itemize}

\textbf{Comportamiento del sistema:}
\begin{itemize}
    \item No existe estado de equilibrio (ningún punto crítico real).
    \item La velocidad terminal es:
    \[
    v = -64 \text{ ft/s}.
    \]
    \item Todas las trayectorias del sistema satisfacen:
    \[
    v(t) \longrightarrow -64 \quad \text{(convergencia asintótica)}.
    \]
\end{itemize}

\subsection*{Diagrama de Fase}

El diagrama de fase se basa en:
\begin{itemize}
    \item La línea nula horizontal $v = -64$ (velocidad terminal).
    \item La línea nula vertical $v = 0$.
\end{itemize}

Las trayectorias se desplazan verticalmente hacia abajo hasta aproximarse asintóticamente a $v=-64$.

\begin{center}
    \vspace{-0.3cm}
    \includegraphics[width=0.5\textwidth]{output1.png}
    \vspace{-0.2cm}
    \captionof{figure}{Diagrama de Fase.}
\end{center}

%===================================================================================




\label{end}

\end{document}

%===================================================================================
